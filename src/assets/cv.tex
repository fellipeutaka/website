\documentclass[10pt, letterpaper]{article}

% Packages:
\usepackage[
	ignoreheadfoot, % set margins without considering header and footer
	top=1 cm, % seperation between body and page edge from the top
	bottom=1 cm, % seperation between body and page edge from the bottom
	left=1 cm, % seperation between body and page edge from the left
	right=1 cm, % seperation between body and page edge from the right
	footskip=1 cm, % seperation between body and footer
	% showframe % for debugging
]{geometry} % for adjusting page geometry
\usepackage{titlesec} % for customizing section titles
\usepackage[dvipsnames]{xcolor} % for coloring text
\definecolor{primaryColor}{RGB}{0, 0, 0} % define primary color
\usepackage{enumitem} % for customizing lists
\usepackage{etoolbox} % for conditional statements
\usepackage{amsmath} % for math
\usepackage[
	pdftitle={Fellipe Utaka's CV},
	pdfauthor={Fellipe Utaka},
	pdfcreator={Fellipe Utaka},
	colorlinks=true,
	urlcolor=primaryColor
]{hyperref} % for links, metadata and bookmarks
\usepackage[pscoord]{eso-pic} % for floating text on the page
\usepackage{calc} % for calculating lengths
\usepackage{bookmark} % for bookmarks
\usepackage{lastpage} % for getting the total number of pages
\usepackage{changepage} % for one column entries (adjustwidth environment)
\usepackage{paracol} % for two and three column entries
\usepackage{needspace} % for avoiding page brake right after the section title
\usepackage{iftex} % check if engine is pdflatex, xetex or luatex

% Ensure that generate pdf is machine readable/ATS parsable:
\ifPDFTeX
\input{glyphtounicode}
\pdfgentounicode=1
\usepackage[T1]{fontenc}
\usepackage[utf8]{inputenc}
\usepackage{lmodern}
\fi

\usepackage{charter}

% Some settings:
\raggedright
\AtBeginEnvironment{adjustwidth}{\partopsep0pt} % remove space before adjustwidth environment
\pagestyle{empty} % no header or footer
\setcounter{secnumdepth}{0} % no section numbering
\setlength{\parindent}{0pt} % no indentation
\setlength{\topskip}{0pt} % no top skip
\setlength{\columnsep}{0.15cm} % set column seperation
\pagenumbering{gobble} % no page numbering

\titleformat{\section}{\needspace{4\baselineskip}\bfseries\large}{}{0pt}{}[
\vspace{1pt}
\titlerule]

\titlespacing{\section}{
% left space:
-1pt }{
% top space:
0.3 cm }{
% bottom space:
0.2 cm } % section title spacing

\renewcommand{\labelitemi}{$\vcenter{\hbox{\small$\bullet$}}$} % custom bullet points
\newenvironment{highlights}{ \begin{itemize}[ topsep=0.10 cm, parsep=0.10 cm, partopsep=0pt,
itemsep=0pt, leftmargin=0 cm + 10pt ] }{ \end{itemize} } % new environment for highlights

\newenvironment{highlightsforbulletentries}{ \begin{itemize}[ topsep=0.10 cm,
parsep=0.10 cm, partopsep=0pt, itemsep=0pt, leftmargin=10pt ] }{ \end{itemize} } % new environment for highlights for bullet entries

\newenvironment{onecolentry}{ \begin{adjustwidth}{ 0 cm + 0.00001 cm }{ 0 cm + 0.00001 cm }
}{ \end{adjustwidth} } % new environment for one column entries

\newenvironment{twocolentry}[2][]{ \onecolentry \def\secondColumn{#2} \setcolumnwidth{\fill, 4.5 cm}
\begin{paracol}{2} }{ \switchcolumn \raggedleft \secondColumn \end{paracol}
\endonecolentry } % new environment for two column entries

\newenvironment{threecolentry}[3][]{ \onecolentry \def\thirdColumn{#3} \setcolumnwidth{, \fill, 4.5 cm}
\begin{paracol}{3} {\raggedright #2} \switchcolumn }{ \switchcolumn \raggedleft \thirdColumn
\end{paracol} \endonecolentry } % new environment for three column entries

\newenvironment{header}{
\setlength{\topsep}{0pt}
\par\kern\topsep
\centering
\linespread{1.5} }{ \par\kern\topsep } % new environment for the header

\newcommand{\placelastupdatedtext}{% \placetextbox{<horizontal pos>}{<vertical pos>}{<stuff>}
\AddToShipoutPictureFG*{% Add <stuff> to current page foreground
\put( \LenToUnit{\paperwidth-2 cm-0 cm+0.05cm}, \LenToUnit{\paperheight-1.0 cm} ){\vtop{{\null}\makebox[0pt][c]{ \small\color{gray}\textit{Last updated in September 2024}\hspace{\widthof{Last updated in September 2024}} }}}%
}%
}%

% save the original href command in a new command:
\let\hrefWithoutArrow\href

% new command for external links:

\begin{document}
	\newcommand{\AND}{\unskip \cleaders\copy\ANDbox\hskip\wd\ANDbox \ignorespaces }
	\newsavebox{\ANDbox}
	\sbox{\ANDbox}{$|$}

	\begin{header}
		\fontsize{25 pt}{25 pt}\selectfont Fellipe Utaka

		\vspace{5 pt}

		\normalsize
		\mbox{\hrefWithoutArrow{mailto:fellipeutaka@gmail.com}{fellipeutaka@gmail.com}}%
		\kern 5.0 pt%
		\AND%
		\kern 5.0 pt%
		\mbox{\hrefWithoutArrow{https://fellipeutaka.com}{fellipeutaka.com}}%
		\kern 5.0 pt%
		\AND%
		\kern 5.0 pt%
		\mbox{\hrefWithoutArrow{https://linkedin.com/in/fellipeutaka}{linkedin.com/in/fellipeutaka}}%
		\kern 5.0 pt%
		\AND%
		\kern 5.0 pt%
		\mbox{\hrefWithoutArrow{https://github.com/fellipeutaka}{github.com/fellipeutaka}}%
	\end{header}

	\section{Habilidades}

	Java, Go, JavaScript (ES2015+), TypeScript, HTML,
	Sass, SQL, GraphQL, React, Next.js, Gatsby, Astro,
	Node.js, Tailwind CSS, Styled Components, Framer
	Motion, Electron, Jest, Vitest, Cypress, Playwright,
	Spring Boot, Express, Fastify, Nest.js, React Native,
	Expo, Git, GitHub, Figma, Netlify, Vercel, Heroku,
	Storybook, Docker, Backstage, MCP, Webpack, Babel, esbuild, swc,
	Jira, Trello, Monday, Notion, Slack, Discord

	\section{Experiência}

	\begin{twocolentry}
		{ Ago 2025 - Atual } \textbf{Engenheiro Full-Stack}, HP via Wipro Consultancy
	\end{twocolentry}

	\vspace{0.10 cm}
	\begin{onecolentry}
		\begin{highlights}
			\item Desenvolvi plataformas de desenvolvedor em escala empresarial e serviços backend integrados com IA para a Plataforma Interna da HP (HPIP)
			\item Projetei e implementei um catálogo de serviços e portal de desenvolvedor baseado em Backstage com plugins backend customizados, APIs REST, modelagem hierárquica de entidades e processamento de webhooks do GitHub em tempo real
			\item Aumentei a cobertura de testes automatizados de 47\% para 86\%, melhorando a confiabilidade do código, manutenibilidade e confiança em refatorações em larga escala
			\item Desenvolvi um servidor MCP baseado em Go permitindo acesso seguro de IA a catálogos e documentação internos, integrando OAuth 2.1 (PKCE), validação JWT, sessões Redis e deployments Kubernetes
			\item Trabalhei com Node.js, TypeScript, Go, React, PostgreSQL, Docker e infraestrutura cloud-native, colaborando com equipes globais
		\end{highlights}
	\end{onecolentry}

	\vspace{0.2 cm}

	\begin{twocolentry}
		{ Set 2024 - Ago 2025 } \textbf{Desenvolvedor Front-End}, Blue Company
	\end{twocolentry}

	\vspace{0.10 cm}
	\begin{onecolentry}
		\begin{highlights}
			\item Desenvolvi e mantive aplicações web escaláveis para uma startup de tecnologia em saúde, utilizando React, TypeScript, Next.js e Astro, melhorando a satisfação do usuário em 20\%
			\item Colaborei com gerentes de produto e designers para criar interfaces amigáveis, utilizando Material UI, TailwindCSS e Radix UI
			\item Implementei designs responsivos e componentes acessíveis com React Aria e react-hook-form, garantindo conformidade com os padrões WCAG
			\item Escrevi código limpo e de fácil manutenção e conduzi revisões de código, introduzindo Zod para validação robusta de entrada e melhorando a qualidade do código em 15\%
		\end{highlights}
	\end{onecolentry}

	\vspace{0.2 cm}

	\begin{twocolentry}
		{ Jun 2024 - Set 2024 } \textbf{Desenvolvedor Full-Stack Freelancer}, Divina Terra Atibaia
	\end{twocolentry}

	\vspace{0.10 cm}
	\begin{onecolentry}
		\begin{highlights}
			\item Construí e implantei um painel administrativo com React, TypeScript, TailwindCSS, Next.js e React Aria, reduzindo a sobrecarga operacional em 15\%
			\item Desenvolvi o back-end usando Java, Spring Boot, Postgres, Docker, pipelines de CI/CD e GitHub Actions, melhorando a eficiência da implantação
			\item Entreguei uma landing page responsiva usando Astro e TailwindCSS, aumentando o engajamento de usuários mobile em 10\%
		\end{highlights}
	\end{onecolentry}

	\vspace{0.2 cm}

	\begin{twocolentry}
		{ Jan 2024 - Jun 2024 } \textbf{Desenvolvedor Full-Stack Freelancer}, NextBase
	\end{twocolentry}

	\vspace{0.10 cm}
	\begin{onecolentry}
		\begin{highlights}
			\item Projetei um boilerplate SaaS modular usando React, TypeScript, TailwindCSS, Next.js, shadcn/ui, Supabase e MDX, acelerando os ciclos de desenvolvimento em 30\%
			\item Criei uma arquitetura escalável e reutilizável para agilizar o desenvolvimento futuro de produtos SaaS, aumentando a produtividade da equipe
		\end{highlights}
	\end{onecolentry}

	\vspace{0.2 cm}

	\begin{twocolentry}
		{ Abr 2023 - Out 2023 } \textbf{Desenvolvedor Full-Stack}, Anjun Express
	\end{twocolentry}

	\vspace{0.10 cm}
	\begin{onecolentry}
		\begin{highlights}
			\item Liderei o desenvolvimento e manutenção de sistemas internos para aprimorar operações de transporte e logística, seguindo as melhores práticas
			\item Otimizei recursos críticos, resultando em um aumento de 20\% na velocidade de carregamento das páginas
			\item Estabeleci um Design System abrangente com componentes reutilizáveis e acessíveis, publicado no NPM para adoção mais ampla
			\item Colaborei em inglês com equipes internacionais para integrar soluções de front e back-end de forma contínua
		\end{highlights}
	\end{onecolentry}

	\section{Educação}

	\begin{twocolentry}
		{ Fev 2025 - Presente } \textbf{Fatec Atibaia}, Tecnólogo em Desenvolvimento de Software Multiplataforma
	\end{twocolentry}

	\vspace{0.10 cm}
	\begin{onecolentry}
		\begin{highlights}
			\item \textbf{Disciplinas:} Estrutura de Dados e Algoritmos, Programação de Sistemas,
			Programação Orientada a Objetos, Programação Funcional
		\end{highlights}
	\end{onecolentry}

	\begin{twocolentry}
		{ Jan 2020 - Dez 2022 } \textbf{ETEC Professor Carmine Biagio Tundisi}, Técnico em
		Desenvolvimento de Sistemas
	\end{twocolentry}

	\vspace{0.10 cm}
	\begin{onecolentry}
		\begin{highlights}
			\item \textbf{Disciplinas:} Estrutura de Dados e Algoritmos, Programação de Sistemas,
			Programação Orientada a Objetos, Programação Funcional
		\end{highlights}
	\end{onecolentry}

	\section{Projetos}

	\begin{twocolentry}
		{\href{https://git.new/kanpeki}{git.new/kanpeki}} \textbf{Kanpeki}
	\end{twocolentry}

	\vspace{0.10cm}
	Uma coleção de componentes prontos para serem copiados e colados, inspirados no shadcn/ui, mas aprimorados utilizando
	React Aria da Adobe e Radix UI. Altamente personalizável, com belas
	animações e componentes acessíveis projetados para integrar perfeitamente em
	projetos React modernos.

	\vspace{0.2cm}

	\begin{twocolentry}
		{\href{https://git.new/satoru}{git.new/satoru}} \textbf{Satoru}
	\end{twocolentry}

	\vspace{0.10cm}
	Uma aplicação desktop construída com Rust, Tauri, React 19, TailwindCSS, shadcn/ui,
	e Tanstack Query. Permite que os usuários gerenciem servidores Minecraft, monitorem logs do servidor,
	acompanhem o uso de RAM e acessem várias outras funcionalidades de gerenciamento.

	\section{Idiomas}

	\begin{onecolentry}
		\begin{highlights}
			\item Inglês (C1)
			\item Português (Nativo)
			\item Espanhol (B2)
		\end{highlights}
	\end{onecolentry}

\end{document}
